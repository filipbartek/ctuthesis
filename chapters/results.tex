\newcommand*{\IncludePaper}[2][height=\paperheight-1.5in]{\includepdf[pages=-, pagecommand={}, #1]{#2}}
%\newcommand*{\IncludePaper}[2][height=\paperheight-1.5in]{\includepdf[pages=1, pagecommand={}, #1]{#2}}
%\newcommand*{\IncludePaper}[2][]{}

\newcommand{\ResultSection}{\chapter}

\todo[inline]{Emphasize that all of the solutions use ML for preprocessing.}
\todo[inline]{Mention that the NNs can be swapped---these results show an example of a NN.}
\todo[inline]{Some papers start with an odd page number (recto; right-hand) in the internal page numbering: \cref{sec:results:simple} (with centered page numbers) and \cref{sec:results:npr}.
Papers in \cref{sec:results:selection,sec:results:regularization} start with an even page number (verso; left-hand).
The paper in \cref{sec:results:cautious} does not have page numbers and does not suggest any page orientation.}

\ResultSection{Learning Precedences from Simple Symbol Features}
\label{sec:results:simple}

State-of-the-art \glspl{atper}, including \gls{vampire}, use the \gls{SuperpositionCalculus} as the core of the reasoning procedure.
Inferences of the \gls{SuperpositionCalculus} are restricted by a \gls{sot}.
% MS: my feeling is that a KBO is instantiated with a precedence, but a scheme (KBO vs LPO) is simply chosen/picked/selected (not instantiated)
\Gls{vampire} implements the \acrfull{kbo} and the \gls{lpo}, two popular simplification orderings.
Each of these orderings is parameterized by a \gls{sp}---permutation of the symbols in the signature.\todo{Most of this has been discussed above.}

The standard heuristic methods implemented in \gls{vampire} construct the \gls{precedence} by sorting the symbols by their arity or number of occurrences in the input problem.
In the work presented below \cite{DBLP:conf/cade/Bartek020},
a \gls{precedence} is constructed using a combination of six syntactic symbol features that are easy to compute.
Arity and number of occurrences are two of these features.
This ensures that the system may be trained to represent, besides various other functions of the features, any of the standard heuristics.

Once the system has been trained,
we can obtain a precedence for an input problem in two steps:
\begin{enumerate}
\item A pairwise preference regressor (Elastic-Net or gradient-boosted decision trees) predicts a preference value for each (ordered) pair of distinct symbols.
This value expresses a degree of importance with which the first of the two symbols should precede the second.
\item A greedy algorithm constructs a precedence using the preference values.
The algorithm approximately optimizes the cost of the precedence, that is,
the sum of preference values of symbol pairs that appear in the precedence as subsequences.\todo[color=green]{Is this explanation clear?}
% The measure we are trying to minimize is precedence cost: sum of costs of symbol pairs that are consistent with the precedence.
\end{enumerate}

The preference regressor is trained on runs of Vampire with \gls{kbo} instantiated by random precedences.
The training aims to minimize the time it takes Vampire to solve an input problem.
%measured by the number of iterations of the saturation loop.

The main contribution of this work is a method of using pairwise ranking as a proxy task in \gls{ml} for \gls{atping} (in this case, to train a \gls{sp} recommender).
This approach proved especially useful in my subsequent research
presented in \cref{sec:results:npr,sec:results:selection}.

\IncludePaper{publications/1-simple}

\ResultSection{Neural Precedence Recommender}
\label{sec:results:npr}

A \gls{gnn} can be used to extract task-specific features of various syntactic elements of an input problem,
including the predicate and function symbols.
In the work presented below \cite{DBLP:conf/cade/Bartek021},
I trained a \gls{gnn} to predict a score for each symbol in the signature such that ordering the symbols by their scores yields a good \gls{sp} for \gls{kbo} in Vampire.

%This work capitalizes on casting the task of training a \gls{sp} recommender as a pairwise ranking task.

A straightforward way to determine the best precedence for a given problem involves evaluating all possible precedences (permutations of the signature of the problem).
Such exhaustive evaluation is prohibitively expensive for all but the smallest problems.
On the other hand, we can easily compare an arbitrary pair of precedences.
Following this observation, we train the symbol score predictor on the proxy task of (pairwise) ranking of precedences.\todo{Mention when a precedence is considered better than another.}

In summary, the \gls{sp} recommender is trained on three nested proxy tasks:
\begin{enumerate}
\item Predicting a score of a symbol
\item Predicting a score of a precedence
\item Pairwise ranking of precedences
\end{enumerate}

The experiments presented in the paper demonstrate that a trained precedence recommender outperforms the baseline by more than \pc{4} in the number of solved problems.
This result confirms that the proxy tasks are aligned with the main task of solving as many problems as possible and demonstrates the potential of customizing symbol precedence automatically.

% R1:
%The paper is impressive: performing machine learning for
%automated theorem proving has so far proved to be notoriously hard
%and new well-explained results in this direction are very welcome. Also, the
%level of detail is sufficient for getting a good insight, which
%is not too common for really complex methods.

% R2:
%  This paper does a good job at presenting both superposition theorem proving and GCN learning at an abstract enough
%  level that researchers in both domains can get an idea of what is going on in the other domain. 
%  The process of training the GCN is also clearly described.

%This paper was published at \gls{cade} 2021.
%Besides the improvement in performance,
%the reviewers praised the clear and detailed presentation of the \gls{gnn} that serves as the backend of the recommender.

\IncludePaper{publications/2-npr}

\ResultSection{A GNN-Advised Clause Selection}
\label{sec:results:selection}

In the work presented below \cite{DBLP:conf/lpar/Bartek023},
we tackled clause selection,
arguably the most important heuristic choice point in a saturation-based prover.
We used the \gls{gnn} architecture
that proved useful in precedence recommendation (see \cref{sec:results:npr})
and trained a \gls{nn} that proposes symbol weights to configure the symbol-counting clause selection heuristic~\cite{DBLP:conf/cade/SchulzM16,E-manual}.

For each problem, the \gls{nn} is only evaluated once as a preprocessing step to determine the symbol weights.
This approach allows using the optimized clause selection procedure implemented in Vampire
with very little modification and negligible computational overhead during the proof search.

The trained system increased the performance of a baseline configuration of Vampire by \pc{6.6},
which is more than the increase introduced by AVATAR,
a powerful technique that delegates some propositional sub-tasks of the \gls{fol} proof search to a highly optimized \gls{sat} solver.
%integrates a \gls{sat} solver into the saturation-based proof search.
Manual inspection revealed that
the system learned to
prioritize clauses with symbols that appear in the conjecture (goal-directed guidance),
deprioritize clauses with subformula names,
and prioritize non-ground clauses---clauses with variable occurrences.

\IncludePaper{publications/3-weights}

\ResultSection{Regularization in Spider-Style Strategy Discovery and Schedule Construction}
\label{sec:results:regularization}

In the research presented below \cite{DBLP:conf/ijcar/BartekCS24},
my co-authors and I created a system that automatically generates strong and mutually complementary strategies (configurations) for the \gls{atper} Vampire.
Using approximately 1000 strategies generated by this system,
we constructed strong strategy schedules that generalize well to unseen problems.

A greedy algorithm is at the core of the schedule construction process.
The algorithm is parameterized with several regularization options.
These new variants of the algorithm and an analysis of their regularizing strength
constitute the main contributions presented in the paper.

The data we used to build the schedules (strategies and their performance measurements)
has been published \cite{bartek10814478}.
The dataset is potentially interesting for research on static algorithm scheduling and \gls{as}.

\IncludePaper{publications/4-regularization}

\ResultSection{Cautious Specialization of Strategy Schedules}
\label{sec:results:cautious}

In \cref{sec:results:regularization},
my co-authors and I focused on building a monolithic strategy schedule that performs well on the whole target set of problems.
The performance of the prover can be further increased by dividing the target set of problems into classes
and constructing a specialized schedule for each of the classes.
Each class should be identified by problem features that are easy to compute
so that a branching system of schedules can be used on an unseen problem efficiently.

As long as the classes are relatively homogeneous in performance of the strategies,
the specialization is expected to improve performance on the training problems.
However, aggressive specialization can lead to a decrease in performance on unseen problems.

In the extended abstract presented below \cite{DBLP:conf/paar/BartekC024},
we expanded on the work introduced in \cref{sec:results:regularization}
by investigating the problem of specialization of strategy schedules.
We performed two initial experiments:
\begin{enumerate}
\item First, we compared a hand-tweaked split of the problem space (dividing problems into three classes by the number of atoms) to a random split.
\item Second, we experimented with training a collection of boosting trees to effectively tailor a schedule for an arbitrary input problem.
\end{enumerate}

While the first experiment showed that the choice of problem features to split on may have a large impact on generalization,
the second experiment represents a viable solution to the schedule specialization task as a whole.\todo{Consider rewording.}\todo{Consider discussing possible future work.}

\IncludePaper{publications/5-cautious}
