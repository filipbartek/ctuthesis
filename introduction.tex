\newcommand{\defn}\emph

\chapter{Introduction}

%c) v úvodní části
%i) přehled současného stavu dané vědní problematiky (s odkazy na
%literaturu) a
%ii) cíle disertace,
%d) v těle práce vlastní výsledky doktoranda s odkazy na jeho publikace v členění
%dle písm. f),
%Má-li práce formu komentovaného souboru publikací dle čl. 1, odst.1.3, pak odst. 3.1. písm.
%c) a odst. 3.1 písm. d) jsou nahrazeny spojujícím textem rozsahu alespoň 10 stran.

\todo[inline]{TBA}

\section{Automatic theorem proving}

Many problems can be encoded in \gls{fol}, such as ...
Solving a problem in \gls{fol} amounts to finding a proof that the conjecture necessarily follows from the premises.
A common approach is \defn{refutation-based proving}:
We search for a proof that a trivial contradiction is a logical consequence of the premises conjoined with the negated conjecture.

\subsection{Saturation-based automatic theorem proving}

A saturation-based \gls{atper} searches for a proof iteratively.
In each iteration, one of the unprocessed clauses is selected and moved to the active clause set.
Then, all the possible inferences within the active clause set are performed.

The process can be improved with forward and backward subsumption.

\subsection{Strategies}

A typical \gls{atper} employs several heuristics that guide the proof search.
The heuristics determine, for example, which clause is selected as the given clause, or which inferences are performed.
Furthermore, the heuristics are typically parameterized, as there is no single configuration that works the best for all problems of interest.
A configuration of all heuristics of a \gls{atper} is known as a strategy.

In the context of \gls{atping}, \defn{\gls{aac}} is the task of finding, given an input problem, a strategy that is expected to solve the problem in a short time.
In general, \gls{aac} can be applied to any parameterized solver.
Several general-purpose \gls{aac} systems have been proposed.

\subsection{TPTP}

\section{Machine learning}

Neural networks, \glspl{gnn}

\section{How to combine machine learning with automatic theorem proving}

Black-box view: heuristic optimization

Premise selection

Clause selection

Simplification ordering on terms

\section{Terminology}

\begin{tabularx}{\linewidth}
{ l c >{\raggedright\arraybackslash}X }
\bfseries Term & \bfseries Symbol & \bfseries Alternatives \\
\Midrule
Strategy & $s$ & Configuration, heuristic, protocol \\
Strategy space & $\Theta$ & Configuration space, parameter space, parameter configuration space \\
Problem & $p$ & Instance \\
Solver & $A$ & Algorithm \\
Strategy schedule & $t_s$ & Heuristic schedule \\
\end{tabularx}
