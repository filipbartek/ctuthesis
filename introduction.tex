\newcommand{\defn}\emph

\chapter{Introduction}

%c) v úvodní části
%i) přehled současného stavu dané vědní problematiky (s odkazy na
%literaturu) a
%ii) cíle disertace,
%d) v těle práce vlastní výsledky doktoranda s odkazy na jeho publikace v členění
%dle písm. f),
%Má-li práce formu komentovaného souboru publikací dle čl. 1, odst.1.3, pak odst. 3.1. písm.
%c) a odst. 3.1 písm. d) jsou nahrazeny spojujícím textem rozsahu alespoň 10 stran.

\section{First-order logic}

\Gls{fol}, also known as predicate logic, is a formal language that allows modeling facts and formally reasoning about them.

\section{Automatic theorem proving (ATP)}

Many problems can be defined in \gls{fol}.
Notable examples include problems in mathematics and software verification.
Solving a problem in \gls{fol} amounts to finding a proof that the conjecture necessarily follows from the premises.
A common approach is \defn{refutation-based proving}:
We search for a proof that a trivial contradiction is a logical consequence of the premises conjoined with the negated conjecture.

\Gls{atping} for \gls{fol} has been successfully applied in software verification and advanced mathematics.
Prominent successes include ...\todo{TBA}

\Gls{casc} is a prominent annual \gls{atping} competition.
In the \gls{fof} division, the provers are evaluated on a set of 500\todo{Really?} \gls{fol} theorems.
To win the competition, the prover needs to solve as many theorems as all of its competitors.

\Gls{casc} is closely related to the \gls{tptp} problem library.
A part of the evaluation problems are sampled from \gls{tptp}, and all the problems in \gls{casc} are added into \gls{tptp} after the competition for later reference.
Both \gls{casc} and the \gls{tptp} library use the \gls{tptp} language to define the problems.
The \gls{tptp} language is, in practice, the standard language for the definition of \gls{fol} problems.

\subsection{Saturation-based theorem proving}

A saturation-based \gls{atper} searches for a proof iteratively.
In each iteration, one of the unprocessed clauses is selected and moved to the active clause set.
Then, all the possible inferences within the active clause set are performed.

The process can be improved with forward and backward subsumption.

\subsection{Strategies}

A typical \gls{atper} employs several heuristics that guide the proof search.
The heuristics determine, for example, which clause is selected as the given clause, or which inferences are performed.
Furthermore, the heuristics are typically parameterized, as there is no single configuration that works the best for all problems of interest.
A configuration of all heuristics of a \gls{atper} is known as a strategy.

In the context of \gls{atping}, \defn{\gls{aac}} is the task of finding, given an input problem, a strategy that is expected to solve the problem in a short time.
In general, \gls{aac} can be applied to any parameterized solver.
Several general-purpose \gls{aac} systems have been proposed.

\subsection{Vampire}

\subsection{TPTP problem library}

\section{Machine learning (ML)}

\Gls{ml} automates the process of creating a responsive system with a useful behavior.\todo{Compare with a dictionary definition.}
In supervised \gls{ml}, such a system is trained on a dataset that consists of training examples,
each of which specifies an input to the system and a corresponding desired output (target label).



\subsection{Neural networks}

Since ..., \glspl{ann} have been a prominent paradigm in \gls{ml}.
The rise of \gls{dl} to prominence was precipitated by the success of \gls{cnn} for image classification.


\Glspl{gnn}

\subsection{Algorithm configuration}

\section{Combining ATP with ML}

Black-box view: heuristic optimization

Premise selection

Clause selection

Simplification ordering on terms

\section{State of the art}
