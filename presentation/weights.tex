\begin{frame}[<+->]{Clause selection}
\begin{itemize}
\item Symbol-counting heuristic:
\begin{enumerate}
\item Count symbol and variable occurrences in each passive clause
\item Select the smallest clause
\end{enumerate}
\item Generalized version: Assign a weight to each symbol
\item GNN-based symbol weight recommender
\end{itemize}
\end{frame}

\begin{frame}{\Gls{gcn}}
\centering
\digraph[scale=0.4]{GcnPlain}{
graph [ranksep=0.3];
node [fontsize=14, shape=record, height=0, label=""];
edge [fontsize=14, dir=both];
problem [color=1];
{ rank = same;
c0 [color=2];
c1 [color=2];
}
problem -> c0 [color=1];
problem -> c1 [color=1];
{ rank = same;
t0 [color=3];
t1 [color=3];
}
c0 -> t0 [color=2];
c1 -> t1 [color=3];
{ rank = same;
tfab [color=4];
tfbb [color=4];
}
{ rank = same;
ta [color=4];
tb [color=4];
}
ff [color=5];
fa [color=5];
fb [color=5];
tfab0 [color=6];
tfab1 [color=6];
tfbb0 [color=6];
tfbb1 [color=6];
t0 -> ta [color=4];
t0 -> tb [color=4];
t1 -> tfab [color=4];
t1 -> tfbb [color=4];
tfab -> ff [color=5];
tfab -> tfab0 [color=6];
tfab0 -> tfab1 [color=7];
tfab0 -> ta [color=8];
tfab1 -> tb [color=8];
tfbb -> ff [color=5];
tfbb -> tfbb0 [color=6];
tfbb0 -> tfbb1 [color=7];
tfbb0 -> tb [color=8];
tfbb1 -> tb [color=8];
ta -> fa [color=5];
tb -> fb [color=5];
}
%{\small $$
%h_d^{(l+1)} =
%\sigma \Parentheses*{\sum_{s \in \mathcal{N}_d} \frac{1}{\sqrt{\card{\mathcal{N}_s}} \sqrt{\card{\mathcal{N}_d}}} (W^{(l)} h_s^{(l)} + b^{(l)})}
%$$}
\end{frame}

\begin{frame}{\Gls{rgcn}}
\centering
\digraph[scale=0.4]{GcnRelational}{
graph [ranksep=0.3];
node [fontsize=14, shape=record, height=0, label=""];
edge [fontsize=14, dir=both, arrowtail=empty, colorscheme=set19];
problem [color=1];
{ rank = same;
c0 [color=2];
c1 [color=2];
}
problem -> c0 [color=1];
problem -> c1 [color=1];
{ rank = same;
t0 [color=3];
t1 [color=3];
}
c0 -> t0 [color=2];
c1 -> t1 [color=3];
{ rank = same;
tfab [color=4];
tfbb [color=4];
}
{ rank = same;
ta [color=4];
tb [color=4];
}
ff [color=5];
fa [color=5];
fb [color=5];
tfab0 [color=6];
tfab1 [color=6];
tfbb0 [color=6];
tfbb1 [color=6];
t0 -> ta [color=4];
t0 -> tb [color=4];
t1 -> tfab [color=4];
t1 -> tfbb [color=4];
tfab -> ff [color=5];
tfab -> tfab0 [color=6];
tfab0 -> tfab1 [color=7];
tfab0 -> ta [color=8];
tfab1 -> tb [color=8];
tfbb -> ff [color=5];
tfbb -> tfbb0 [color=6];
tfbb0 -> tfbb1 [color=7];
tfbb0 -> tb [color=8];
tfbb1 -> tb [color=8];
ta -> fa [color=5];
tb -> fb [color=5];
}
%{\small $$
%h_d^{(l+1)} =
%\mathemph{\sum_{r \in \mathcal{R}}} \sigma \Parentheses*{\sum_{s \in \mathcal{N}_d^{\mathemph{r}}} \frac{1}{\sqrt{\card{\mathcal{N}_s^{\mathemph{r}}}} \sqrt{\card{\mathcal{N}_d^{\mathemph{r}}}}} (W_{\mathemph{r}}^{(l)} h_s^{(l)} + b_{\mathemph{r}}^{(l)})}
%$$}
\end{frame}

\begin{frame}
\frametitle{Graph representation of a \acrshort{cnf} problem}
% TODO: Use one running example.
% Mention: relational graph, node types, edge types

Input problem: $a=b \wedge f(a,b) \neq f(b,b)$

\centering
\digraph[scale=0.4]{GcnExample}{
graph [ranksep=0.3];
node [fontsize=14, shape=record, height=0, width=0];
edge [fontsize=14, dir=both, arrowtail=empty, colorscheme=set19];
problem [label=<a=b &and; f(a,b)&ne;f(b,b)|problem>, color=1];
{ rank = same;
c0 [label="a=b|clause", color=2];
c1 [label=<f(a,b)&ne;f(b,b)|clause>, color=2];
}
problem -> c0 [color=1];
problem -> c1 [color=1];
{ rank = same;
t0 [label="a=b|equality atom", color=3];
t1 [label="f(a,b)=f(b,b)|equality atom", color=3];
}
c0 -> t0 [label=" pos ", color=2];
c1 -> t1 [label=" neg ", color=3];
{ rank = same;
tfab [label="f(a,b)|term", color=4];
tfbb [label="f(b,b)|term", color=4];
}
{ rank = same;
ta [label="a|term", color=4];
tb [label="b|term", color=4];
}
ff [label="f|function", color=5];
fa [label="a|function", color=5];
fb [label="b|function", color=5];
tfab0 [label="|argument", color=6];
tfab1 [label="|argument", color=6];
tfbb0 [label="|argument", color=6];
tfbb1 [label="|argument", color=6];
t0 -> ta [color=4];
t0 -> tb [color=4];
t1 -> tfab [color=4];
t1 -> tfbb [color=4];
tfab -> ff [color=5];
tfab -> tfab0 [color=6];
tfab0 -> tfab1 [color=7];
tfab0 -> ta [color=8];
tfab1 -> tb [color=8];
tfbb -> ff [color=5];
tfbb -> tfbb0 [color=6];
tfbb0 -> tfbb1 [color=7];
tfbb0 -> tb [color=8];
tfbb1 -> tb [color=8];
ta -> fa [color=5];
tb -> fb [color=5];
% Technical details:
%ff -> ff [dir=forward];
%fa -> fa [dir=forward];
%fb -> fb [dir=forward];
}
\end{frame}


\begin{frame}{Symbol weight recommender}
%\begin{columns}
%\column{0.7\textwidth}
\centering
\input{presentation/gv/RankNetWeightPredict}
%\column{0.3\textwidth}
%\vspace{0.5cm}
%\tiny $$\softplus(x) = \log(1 + e^x)$$
%\vspace{1.5cm}
%\tiny $$\sigma(x) = \frac{1}{1 + e^{-x}}$$
%\end{columns}

%\centering
% TODO: Improve the diagram: make it prettier, use colors etc.
%\input{presentation/gv/RankNetWeightPredict}
%\digraph[scale=0.45]{TrainingEvaluation}{
%graph [splines=ortho, ranksep=0.25];
%node [shape=box, fontsize=14, width=0, height=0];
%Problem [label="Input problem", pos="0,0!"];
%subgraph cluster_recommender {
%	label="Recommender";
%	style=dashed;
%	GCN [style=<rounded>, label="Neural network", pos="0,10!"];
%	Activation [style=<rounded>, label=<Output activation<br/><i>1+softplus(*)</i>>];
%}
%SymbolCost [label=<Symbol weights <i>w(*)</i>>];
%Problem -> GCN -> Activation -> SymbolCost;
%%
%subgraph cluster_evaluation {
%	label="Evaluation";
%	style=dashed;
%	ATP [style=<rounded>, label=<Automatic theorem prover>];
%	Result [label=<Result>];
%	ATP -> Result;
%}
%Problem -> ATP;
%SymbolCost -> ATP;
%%
%subgraph cluster_training {
%	label="Training";
%	style=dashed;
%	{ rank = same;
%	c1 [label=<Clause <i>C<sub>+</sub></i>>];
%	c0 [label=<Clause <i>C<sub>&minus;</sub></i>>];
%	}
%	{ rank = same;
%	wc1 [label=<Clause weight <i>W(C<sub>+</sub>)</i>>];
%	wc0 [label=<Clause weight <i>W(C<sub>&minus;</sub>)</i>>];
%	}
%	c1 -> wc1;
%	c0 -> wc0;
%	Logit [label=<Logit <i>W(C<sub>&minus;</sub>) &minus; W(C<sub>+</sub>)</i>>];
%	wc1 -> Logit;
%	wc0 -> Logit;
%%	Prob [label=<Likelihood <i>p = sigmoid(W(C<sub>&minus;</sub>) &minus; W(C<sub>+</sub>))</i>>];
%	Loss [label=<Loss <i>l = &minus; log sigmoid(W(C<sub>&minus;</sub>) &minus; W(C<sub>+</sub>))</i>>];
%%	Logit -> Prob -> Loss;
%	Logit -> Loss;
%}
%SymbolCost -> wc1;
%SymbolCost -> wc0;
%SymbolCost -> c1 [style="invis"];
%SymbolCost -> c0 [style="invis"];
%}
\note{This is an instance of RankNet. We are learning symbol weights to rank clauses. The clause ranking task is a mere proxy task.

The scale of influence of the recommender is small, since it does not evaluate the passive clauses themselves, but only the symbol weights.
This is compensated by the computational cheapness:
The recommender is only evaluated once in the beginning of the proof search,
and there is virtually no overhead over standard weight-based selection.

Note the output activation function.
It ensures a positive lower bound on the symbol weights.
This in turn ensures fairness of the selection.
Negative symbol weights tend to lead to useless attractors.}
\end{frame}

\begin{frame}{Summary: Symbol weight recommender}
\begin{itemize}
\item Strengths:
\begin{itemize}
\item One evaluation of GNN per proof search
\item Negligible computational overhead during proof search
\item Signature-agnostic
\end{itemize}
\item Empirical result: +\pc{6.5620542} problems solved
\end{itemize}
\end{frame}

\begin{frame}[<+->]{Brief mention: Neural Precedence Recommender}
\begin{itemize}
\item Superposition calculus is restricted by symbol precedence
\item GNN-based precedence recommender:
\begin{itemize}
\item Trainable from pairs of precedences (learning to rank)
\item Easy construction of the recommended precedence (sort symbols by predicted costs)
%\item To train by RankNet, we define precedence cost to be minimized by the good precedences.
%\item To allow easy construction of a good precedence, we define the precedence cost as a linear combination of symbol costs (dot product of symbol costs and symbol ranks in the precedence). Then sorting symbols by cost minimizes precedence cost.
%\item Normalization: Precedence cost is weighted by $\frac{2}{n(n+1)}$
\end{itemize}
\end{itemize}
\end{frame}
