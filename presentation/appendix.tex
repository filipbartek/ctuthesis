{\usebackgroundtemplate{\titlebackground}
\frame[plain]{
\vfill
\centering
\color{\titlecolor} \bfseries \Large Backup slides
\vfill
}
}

\section{Publications}
\begin{frame}{Publications}
\begin{itemize}
\item \underline{Filip Bártek} and Martin Suda.\\Learning Precedences from Simple Symbol Features. PAAR 2020.
\item \underline{Filip Bártek} and Martin Suda.\\Neural Precedence Recommender. CADE 2021.\footnote[1]{CORE conference rank: A}
\item \underline{Filip Bártek} and Martin Suda.\\How Much Should This Symbol Weigh? A GNN-Advised Clause Selection. LPAR~2023.\footnotemark[1]
\item \underline{Filip Bártek}, Karel Chvalovský, and Martin Suda.\\Regularization in {S}pider-Style Strategy Discovery and Schedule Construction. IJCAR~2024.\footnotemark[1]
\item \underline{Filip Bártek}, Karel Chvalovský, and Martin Suda.\\Cautious Specialization of Strategy Schedules (Extended Abstract). PAAR 2024.
\end{itemize}
\end{frame}

\section{Prerequisites}

\subsection{First-order logic}

\begin{frame}{First-order logic (FOL) with equality}
\begin{itemize}
\item Logical symbols
\begin{itemize}
\item Connectives: $\land$, $\lor$, $\lnot$, $\to$
\item Quantifiers: $\forall$, $\exists$
\item Variables: $x$, $y$, etc.
\item Equality: $=$
\end{itemize}
\item Non-logical symbols
\begin{itemize}
\item Predicates: $P$, $Q$, etc.
\item Functions (including constants): $+$, $s$, $0$, $f$, $g$, etc.
\end{itemize}
\end{itemize}

In this work, we use the \defn{clause normal form (CNF)}.
\end{frame}

%\begin{frame}{FOL CNF problem}
%Signature: Set of all predicate and function symbols
%
%Input problem: Set of clauses
%\end{frame}

%\subsection{TPTP problem library}
%
%\begin{frame}{TPTP problem library}
%Many problems from various domains in FOL and CNF
%\end{frame}

\subsection{Saturation-based theorem proving}

\begin{frame}{Saturation-based theorem proving}

Input: Set of clauses
\bigskip

Proof search state -- two sets of clauses:
\begin{itemize}
\item Passive
\item Active
\end{itemize}
\bigskip

Saturation loop:
\begin{enumerate}
\item Select clause $C$ from Passive.
\item Move $C$ from Passive to Active.
\item Perform all inferences in Active in which $C$ participates.
\begin{itemize}
%\item Prune the inferences by \emph{literal selection} and \emph{equation orienting}
\item Add the generated clauses to Passive.
\item If the empty clause $\square$ is generated, terminate.
\end{itemize}
\end{enumerate}
\end{frame}

%\begin{frame}{Saturation-based theorem proving}
%Repeatedly infer clauses provable from the input problem.
%
%When the empty clause (trivial contradiction) is inferred, we conclude that the conjecture holds.
%
%When no new clauses can be inferred, we conclude that the conjecture does not hold.
%
%When the conjecture does not hold, the proof search may run forever.
%\end{frame}

\section{Learning symbol precedences for simplification ordering on terms}

\begin{frame}{Simplification ordering on terms}
\begin{itemize}
\item Superposition calculus is parameterized by \defn{simplification ordering on terms}
\item Prunes the proof search without compromising completeness:
\begin{itemize}
\item Literal selection
\item Equation orienting
\end{itemize}
\item Knuth-Bendix ordering is parameterized by a \emph{symbol precedence} -- a permutation of the signature of the input problem
\begin{itemize}
\item Standard heuristics: sort the symbols by arity or number of occurrences
\end{itemize}
\end{itemize}

Learning task: Given an input problem, recommend a symbol precedence.
\end{frame}

\begin{frame}{Neural Precedence Recommender}
\centering
\input{presentation/gv/RankNetPrecedence}
%\digraph[scale=0.4]{PresentationArchitectureOverview}{
%graph [splines=ortho, ranksep=0.25];
%node [shape=box, fontsize=14, width=0, height=0];
%Problem [label="First-order logic problem"];
%Graphifier [style=rounded, label="Graph constructor"];
%g [label="Graph"];
%GCN [style=<rounded,bold>, label="Graph convolutional network"];
%SymbolEmbedding [label="Symbol embeddings"];
%SymbolCostModel [style=<rounded,bold>, label="Output layer"];
%SymbolCost [label="Symbol costs"];
%Problem -> Graphifier -> g -> GCN;
%GCN -> SymbolEmbedding -> SymbolCostModel -> SymbolCost [penwidth=3];
%subgraph cluster_prediction {
%	label="Prediction";
%	style=dashed;
%	Sort [style=rounded, label="Sort"];
%	Precedence [label=<Recommended precedence>];
%	Sort -> Precedence;
%}
%SymbolCost -> Sort;
%subgraph cluster_training {
%	label="Training";
%	style=dashed;
%	{ rank = same;
%	pi [label=<Precedence &pi;>];
%	rho [label=<Precedence &rho;>];
%	}
%	LossFunction [style=rounded, label="Loss function"];
%	Loss [label="Loss value"];
%	pi -> LossFunction:nw;
%	rho -> LossFunction:ne;
%	LossFunction -> Loss [penwidth=3];
%}
%SymbolCost -> LossFunction [penwidth=3];
%}
\end{frame}

\begin{frame}
\frametitle{Training: Precedence cost}
$c_i$ is the cost of the $i$-th symbol.

\begin{block}{Cost of symbol precedence $\pi$ over signature of length $n$}
\begin{equation*}
C(\pi) = \frac{2}{n(n+1)} \sum_{i=1}^n i \cdot c_{\pi(i)}
\end{equation*}
\end{block}

\begin{lemma}[Precedence cost minimization]
The precedence cost $C$ is minimized by any precedence that sorts the symbols by their costs in non-increasing order:
$$
\argmin_{\rho \in \mathrm{Perm}(n)} C(\rho) = \argsort^- (c_1, \ldots, c_n)
$$
\end{lemma}

\end{frame}

\begin{frame}
\frametitle{Training: Loss function}


\begin{block}{Loss on training example $\Better{\PrecBetter}{\PrecWorse}{P}$}
Precedence $\PrecBetter$ is better than precedence $\PrecWorse$ for problem $P$.

$C(\pi)$ is the predicted cost of precedence $\pi$.

\begin{equation*}
\loss(P, \PrecBetter, \PrecWorse) = - \log \sigmoid (C(\PrecWorse) - C(\PrecBetter))
\end{equation*}

\centering
% https://pgf-tikz.github.io/pgf/pgfmanual.pdf : 94. Mathematical Expressions
\begin{tikzpicture}
\begin{axis}[
    domain=-12:12,
    xmin=-12,
    xmax=12,
    ymin=-1,
    ymax=11,
    samples=100,
    xlabel={$C(\PrecWorse) - C(\PrecBetter)$},
    ylabel={$\loss(P, \PrecBetter, \PrecWorse)$},
    scale only axis,
    width=0.25\textwidth,
    height=0.125\textwidth
]
%\addplot[gray] {0};
%\addplot[gray] {-x};
\addplot[] {-ln(\FuncSigmoid(x))};
\end{axis}
\end{tikzpicture}
\end{block}

\end{frame}

\begin{frame}
\frametitle{Evaluation}

\fontsize{10pt}{12}\selectfont

\centering
\begin{tabular}{l|ll|rl}

Symbol cost model & \multicolumn{2}{l}{Success count\footnote{Total number of validation problems: \num{7648}. Number of repetitions: 5.}} & \multicolumn{2}{l}{Improvement} \\
& Mean & Std & Absolute & Relative \\

\hline


\acrshort{gcn} (pred. and func.) &
% VML-706
\num{3951.6} &
\num[round-mode=places,round-precision=2]{1.624807680927192} &
%\SI{51.69}{\percent} &
+182.0 &
\num[round-mode=places,round-precision=3]{1.048280985} \\


\acrshort{gcn} (predicate only) &
% Success rate: mean: 0.513023013, std: 0.000292887
% Evaluation results: https://ui.neptune.ai/filipbartek/vampire-ml/e/VML-553
% Total: validation_solver_eval/all/problems/measured&split&category: 7648
% Difference in success count from baseline: 154 ~ 0.020135983
% Estimated difference from baseline (estimate on 891 problems): 0.021099888
% Checkpoint: outputs/2021-02-06/14-55-41/tf_ckpts/epoch/weights.00079-0.61.tf VML-540 0.511785
\num{3923.6} &
% Success mean: validation_solver_eval/all/success/count/mean: 3923.6
\num{2.24} &
% Success std: validation_solver_eval/all/success/count/std 2.24
%\SI{51.30}{\percent} &
% Success rate: 0.513023013
+154.0 &
\num[round-mode=places,round-precision=3]{1.040853141} \\


\acrshort{gcn} (function only) &
% Final evaluation: VML-677
% Evaluated checkpoint: outputs/2021-02-16/12-28-14/tf_ckpts/epoch/weights.00289.tf
% Results file: sftp://cluster.ciirc.cvut.cz/home/bartefil/git/vampire-ml/outputs/2021-02-17/12-01-09/solver_eval/symbol_cost/epoch_-1/logs.yaml
% Total: val/all/problems/measured&split&category: 7648
\num{3874.2} &
% Success mean: val/all/success/count/mean: 3874.2
\num[round-mode=places,round-precision=2]{1.8330302779823362} &
% Success std: val/all/success/count/mean: 1.8330302779823362
%\SI{50.66}{\percent} &
% Success rate: val/all/success/count/mean: 0.5065638075313807
+104.6 &
\num[round-mode=places,round-precision=3]{1.027748302} \\


%Simple (predicate only) &
% https://ui.neptune.ai/filipbartek/vampire-ml/e/VML-737
%\num{3827.2} &
%\num[round-mode=places,round-precision=2]{1.9390719429665317} &
%\SI{50.04}{\percent} &
%+57.6 &
%\num[round-mode=places,round-precision=3]{1.015280136} \\
% Final vector from PAAR paper: [0,0.429306921481749,0.57069307851825,0,0,0,0,0,0,0,0,0]
% Source: https://docs.google.com/spreadsheets/d/1HSsC7piUAtWt6uwA9SYOX5vmiF0Ab3Ae4FPyGsDALIg/edit#gid=99404775


%Frequency (regular \acrshort{kbo}) &
%\texttt{vampire -lcm standard} &
%\num{3823.0} &
%\num[round-mode=places,round-precision=2]{3.40587727318528} &
%\SI[round-mode=places,round-precision=2]{49.9869247}{\percent} &
%+53.4 &
%\num[round-mode=places,round-precision=3]{1.014165959} \\
% VML-741


Frequency (baseline) &
%\texttt{vampire -lcm predicate} &
% Success rate: mean: 0.492887029, std: 0.000401412
% https://ui.neptune.ai/filipbartek/vampire-ml/e/VML-490
% sftp://cluster.ciirc.cvut.cz/home/bartefil/git/vampire-ml/outputs/2021-02-04/17-17-38
% /home/filip/projects/vampire-ml/vampire-ml/outputs/2021-02-09/12-13-43
% Row: 'val&graphified&solver_eval'
% Problems total: 7648
% Success rate: 0.492887029
\num{3769.6} &
\num{3.07} &
%\SI{49.29}{\percent} &
0.0 &
\num[round-mode=places,round-precision=3]{1.0} \\

\end{tabular}

\end{frame}

\begin{frame}
\frametitle{Summary}
\begin{itemize}
\item \Acrshort{gcn} over a graph representation of a \acrshort{fol} problem predicts symbol costs
\item Sorting symbols by symbol costs yields a precedence
\item Training is performed on oriented precedence pairs ``$\Better{\PrecBetter}{\PrecWorse}{P}$'' \\
(``Precedence $\PrecBetter$ is better than precedence $\PrecWorse$ in problem $P$.'')
\item Combination of two \acrshortpl{gcn} outperforms the ``frequency'' heuristic by \SI{4.8}{\percent}
\end{itemize}
\end{frame}


\section{Clause selection}
\begin{frame}{Evaluation}

% TODO: This slide is too complex. Simplify.

\begin{center}
\begin{tabular}{l|rrrrr}
% Formatting inpsiration:
% - Vampire with a brain ...: Table 2
% - CASC-J11 report
% - Neural Precedence Recommender
%\hline

Configuration & \multicolumn{2}{c}{Proofs found} & \multicolumn{3}{c}{Compared to B} \\
& /3149 & \% & $+$ & $-$ & \% \\

\hline

Trained \acrshort{gnn} & \num{1494} & \pc{47.4436} & +\num{141} & \num{-49} & +\pc{6.5620542} \\
% Evaluation results: /mnt/cluster/home/bartefil/git/vampire-ml/weight/workspace/nwc/outputs/eval/champion/epoch/-1/eval/problems.pkl
% Success filter: dataset_train=false & dataset_val=false & valid=true & success_uns=true & (megainstructions is null | megainstructions <= 50000)
% `megainstructions is null` counts as success because it represents a run with 0 megainstructions.

% FB: I accidentally used the figure 1439 in the reviewed version.
% 1439 is the number of successes in the test set before large problems were removed, so within 3487 (rather than 3149) problems.
% The originally reported percentage (44.5221) was for the correct success count (1402).
Baseline (B) & \num{1402} & \pc{44.5221} & +0 & $-0$ & +\pc{0.0} \\

\hline

%\Gls{gnn} trained on nwc=5 & \num{1485} & & & & \\
% Evaluation results: /mnt/cluster/home/bartefil/git/vampire-ml/weight/workspace/nwc/outputs/eval/nwc5-large-test/epoch/-1/eval/problems.pkl

B + AVATAR & \num{1485} & \pc{47.1578279} & & & +\pc{5.920114123} \\
B + Goal-directed & \num{1463} & \pc{46.4592} & & & +\pc{4.350927247} \\

% TODO: Include paths to data in comments.

%\hline
\end{tabular}
\end{center}

\note{The dashed line shows the default evaluation limit of \num{5e10} instructions
that was used to generate the training data.
The configurations were evaluated to up to \num{20e10} instructions.

In our experimental setup, \num{5e10} instructions were executed in approximately 16 seconds of wallclock time.

Results reported are for \num{5e10} instructions.}
\end{frame}

\section{Strategies}

\begin{frame}{Strategy discovery}
\note{The discovery is inspired by Spider. Spider was invented by Andrei Voronkov.}
	% TODO: Use algorithmic package to typeset nice pseudocode
	\note{We run the strategy collection in 3 phases: "unsolved", "improving", and "group".}
    Repeat:
	\begin{enumerate}
		\item Sample strategy $s$, time limit $t$, problem $p$
        \note{The options are sampled nearly independently. The only dependencies turn the option on or off. The distribution is biased in favor of complementarily-strong strategies (based on a previous strategy discovery experiment).}
        \note{96 Vampire options (categorical and numeric) + 7 auxiliary parameters (categorical)}
		\note{We set \texttt{sil} (simulated instruction limit) to $2t$ because the evaluation will run with the limit of $2t$.}
		% Unsolved: From unsolved problems biased in favor of problems with low TPTP rating
		% Improving: Uniformly from problems whose optimum solving time is between 1000 and 2000 (if $t = 2000$) - in a bracket
		% If t=1000, the lower bound of the bracket is some constant smaller than 1000.
		% We also have 3rd approach: sample a group of problems
		\item Attempt to solve $p$ with $s$ in time limit $t$. If success:
		% Unsolved: solving is success
		% Improving: improving the solving time to one half or better is success
		\begin{itemize}
			\item Optimize $s$ on $p$ by local search to reduce runtime
			\note{Local search: 2 rounds of free local search (sampling using a distribution), 3 rounds of minimization (only steps towards default strategy). Each round perturbs each parameter once.}
			\item Evaluate $s$ on all problems
			%\item Store $s$ and the evaluation results
		\end{itemize}
	\end{enumerate}
\end{frame}

\begin{frame}{Strategy schedule construction}
\begin{block}{Input}
\begin{itemize}
\item Strategies $S$
\note{We optimize \emph{strategy} schedules but our technique applies to arbitrary algorithm schedules.}
\item Problems $P$
\item Runtime measurements $E : P \times S \to \NatExt$
\note{Runtime is expressed in Mi.
$\infty$ represents timeout or other failure.
Runtime measurement values are extended natural numbers.
We assume determinism.}
\item Runtime budget $T \in \Nat$
\end{itemize}
\end{block}
\begin{block}{Output}
Schedule $\sched{s} : S \to \Nat$ such that $\sum_{s \in S} \sched{s} \leq T$
\note{The schedule is sequential and unordered.}
\end{block}
\begin{block}{Maximize}
%Maximize the number of solved problems:
$$\norm*{\bigcup_{s \in S} \SetBuilder{p \in P}{E(p, s) \leq \sched{s}}}$$
%\CountBuilder{p \in P}{\exists s \in S : E(p, s) \leq \sched{s}}
%$$\sum_{p \in P} \iverson{\bigvee_{s \in S} E(p, s) < \sched{s}}$$
% TODO: Stress the meaning of schedule performance. Show an example.
\note{Other possible criteria: \gls{par} score}
\end{block}
\note{In practice, the schedule is typically run sequentially. We disregard the order of slices in this work.}
% TODO: How many slices do our schedules have?
\end{frame}

\begin{frame}{Generalizing to unseen problems}
For a fixed time budget $T$, repeat:
\note{$T \in \{16000, 64000, 256000\}$}
\begin{enumerate}
\item Randomly split $P$ into $\ProblemsTrain$ and $\ProblemsTest$ (80:20)
%\item Training problems: $|\ProblemsTrain| \approx \SI{80}{\percent} \cdot |P| \approx 6293$
\note{$\norm*{\ProblemsTrain} \approx 7866 \cdot \frac{4}{5} = 6292.8$}
%\item Strategies: $|S| \approx 829$
\note{$|\StrategiesTrain| \approx 1 + 1035 \cdot \SI{80}{\percent} = 829$}
\item Construct schedule $\sched{s}$ using runtime measurements on $\ProblemsTrain$
\item Evaluate $\sched{s}$ using $\ProblemsTest$: How many problems does the schedule solve in runtime $T$?
\note{We use simulated evaluation using $E$.}
\end{enumerate}
Average the evaluation results.
\end{frame}

\begin{frame}{Regularization methods}
% Describe briefly each method. Talk about some motivation.
% alpha: arbitrary stressing
% beta: redundance

% Intuition: We don't know a priori how much we should stress solved problems or time. ... We may be fitting to our data.

\begin{table}
\centering
%\caption{Regularization methods}
\begin{tabular}{@{}lll@{}}
\toprule
Regularization method & Parameter & Default \\
\midrule
Additive slack & $\textcolor{b}{b} \in \Nat$ & $b = 0$ \\
Multiplicative slack & $\textcolor{w}{w} \geq 1$ & $w = 1$ \\
Diminishing problem rewards & $0 \leq \textcolor{beta}{\beta} \leq 1$ (discount factor) & $\beta = 0$ \note{Motivation for $\beta$: We want to cover the problems robustly.} \\
Temporal reward adjustment & $0 \leq \textcolor{alpha}{\alpha}$ (reward exponent) & $\alpha = 1$ \note{The relative stress on number of problems solved and time is arbitrary. We expose this tradeoff.} \\
\bottomrule
\end{tabular}
\end{table}

\begin{block}{Slice extension criterion with reward adjustment}
\begin{algorithmic}
\State $s, t \gets \argmax_{s \in S, 0 < t \leq T'} \frac{\RoundBracket*{\sum_{p \in P} \iverson{\sched{s} < \SolveTimeP{p}{s} \leq \sched{s} + t} \textcolor{beta}{\beta^{\mli{\#covered}(p)}}}^{\textcolor{alpha}{\alpha}}}{t}$
\end{algorithmic}
\end{block}

\begin{block}{Schedule post-processing with slack}
\begin{algorithmic}
\ForAll{$s \in S$ such that $\sched{s} > 0$}
\State $\sched{s} \gets \sched{s} \cdot \textcolor{w}{w} + \textcolor{b}{b}$
\EndFor
\end{algorithmic}
\end{block}
\end{frame}

\begin{frame}{Temporal reward adjustment ($\alpha = 1.7$)}
\only<1>{\includegraphics{figure/presentation/greedy/1}}
\only<2>{\includegraphics{figure/presentation/greedy/17}}
\note{The curve is $t^{\frac{1}{\alpha}} \cdot c$.}
% score == best.solved_cumcount ** tra / best.runtime_increment
% solved_cumcount == (runtime_increment * score) ** (1 / tra)
\end{frame}

\newcommand{\perf}{\num[round-mode=places,round-precision=0]}
\newcommand{\ttf}{\num[round-mode=places,round-precision=1]}
\begin{frame}{Empirical results}
\begin{table}[t]
    \centering
    \caption{Budget: \Mi{64000}.
    Performance is the mean number of problems solved out of \num{7866} across 50 splits.
    Time to fit is the mean time to construct a schedule in seconds.}
    \begin{tabular}{@{}c|rr|r@{}}
        \toprule
        \multirow{2}{*}{Regularization} & \multicolumn{2}{c}{Performance} & \multirow{2}{*}{Time to fit [s]} \\
        & Test & Train & \\
        \midrule

$\alpha = 1.7$ & \emph{\perf{5704.399593}} & \perf{5904.724703} & \ttf{ 4.325128} \\
$\beta = 0.3$  & \perf{5641.199691} & \perf{5955.299844} & \ttf{66.538374} \\
$w = 1.1$      & \perf{5625.900160} & \perf{5946.099763} & \ttf{19.332677} \\
$b = 10$       & \perf{5620.700770} & \perf{5971.249778} & \ttf{19.607397} \\
None (default) & \perf{5616.100439} & \emph{\perf{5986.124790}} & \ttf{19.885694} \\

        \bottomrule
    \end{tabular}
\end{table}
\end{frame}

\section{Acknowledgments}
\begin{frame}{Acknowledgments}
\Acknowledgments
\end{frame}
