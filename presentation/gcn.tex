\begin{frame}{\Gls{gcn}}
\centering
\digraph[scale=0.4]{GcnPlain}{
graph [ranksep=0.3];
node [fontsize=14, shape=record, height=0, label=""];
edge [fontsize=14, dir=both];
problem [color=1];
{ rank = same;
c0 [color=2];
c1 [color=2];
}
problem -> c0 [color=1];
problem -> c1 [color=1];
{ rank = same;
t0 [color=3];
t1 [color=3];
}
c0 -> t0 [color=2];
c1 -> t1 [color=3];
{ rank = same;
tfab [color=4];
tfbb [color=4];
}
{ rank = same;
ta [color=4];
tb [color=4];
}
ff [color=5];
fa [color=5];
fb [color=5];
tfab0 [color=6];
tfab1 [color=6];
tfbb0 [color=6];
tfbb1 [color=6];
t0 -> ta [color=4];
t0 -> tb [color=4];
t1 -> tfab [color=4];
t1 -> tfbb [color=4];
tfab -> ff [color=5];
tfab -> tfab0 [color=6];
tfab0 -> tfab1 [color=7];
tfab0 -> ta [color=8];
tfab1 -> tb [color=8];
tfbb -> ff [color=5];
tfbb -> tfbb0 [color=6];
tfbb0 -> tfbb1 [color=7];
tfbb0 -> tb [color=8];
tfbb1 -> tb [color=8];
ta -> fa [color=5];
tb -> fb [color=5];
}
%{\small $$
%h_d^{(l+1)} =
%\sigma \Parentheses*{\sum_{s \in \mathcal{N}_d} \frac{1}{\sqrt{\card{\mathcal{N}_s}} \sqrt{\card{\mathcal{N}_d}}} (W^{(l)} h_s^{(l)} + b^{(l)})}
%$$}
\end{frame}

\begin{frame}{\Gls{rgcn}}
\centering
\digraph[scale=0.4]{GcnRelational}{
graph [ranksep=0.3];
node [fontsize=14, shape=record, height=0, label=""];
edge [fontsize=14, dir=both, arrowtail=empty, colorscheme=set19];
problem [color=1];
{ rank = same;
c0 [color=2];
c1 [color=2];
}
problem -> c0 [color=1];
problem -> c1 [color=1];
{ rank = same;
t0 [color=3];
t1 [color=3];
}
c0 -> t0 [color=2];
c1 -> t1 [color=3];
{ rank = same;
tfab [color=4];
tfbb [color=4];
}
{ rank = same;
ta [color=4];
tb [color=4];
}
ff [color=5];
fa [color=5];
fb [color=5];
tfab0 [color=6];
tfab1 [color=6];
tfbb0 [color=6];
tfbb1 [color=6];
t0 -> ta [color=4];
t0 -> tb [color=4];
t1 -> tfab [color=4];
t1 -> tfbb [color=4];
tfab -> ff [color=5];
tfab -> tfab0 [color=6];
tfab0 -> tfab1 [color=7];
tfab0 -> ta [color=8];
tfab1 -> tb [color=8];
tfbb -> ff [color=5];
tfbb -> tfbb0 [color=6];
tfbb0 -> tfbb1 [color=7];
tfbb0 -> tb [color=8];
tfbb1 -> tb [color=8];
ta -> fa [color=5];
tb -> fb [color=5];
}
%{\small $$
%h_d^{(l+1)} =
%\mathemph{\sum_{r \in \mathcal{R}}} \sigma \Parentheses*{\sum_{s \in \mathcal{N}_d^{\mathemph{r}}} \frac{1}{\sqrt{\card{\mathcal{N}_s^{\mathemph{r}}}} \sqrt{\card{\mathcal{N}_d^{\mathemph{r}}}}} (W_{\mathemph{r}}^{(l)} h_s^{(l)} + b_{\mathemph{r}}^{(l)})}
%$$}
\end{frame}

\begin{frame}
\frametitle{Graph representation of a \acrshort{cnf} problem}
% TODO: Use one running example.
% Mention: relational graph, node types, edge types

Input problem: $a=b \wedge f(a,b) \neq f(b,b)$

\centering
\digraph[scale=0.4]{GcnExample}{
graph [ranksep=0.3];
node [fontsize=14, shape=record, height=0, width=0];
edge [fontsize=14, dir=both, arrowtail=empty, colorscheme=set19];
problem [label=<a=b &and; f(a,b)&ne;f(b,b)|problem>, color=1];
{ rank = same;
c0 [label="a=b|clause", color=2];
c1 [label=<f(a,b)&ne;f(b,b)|clause>, color=2];
}
problem -> c0 [color=1];
problem -> c1 [color=1];
{ rank = same;
t0 [label="a=b|equality atom", color=3];
t1 [label="f(a,b)=f(b,b)|equality atom", color=3];
}
c0 -> t0 [label=" pos ", color=2];
c1 -> t1 [label=" neg ", color=3];
{ rank = same;
tfab [label="f(a,b)|term", color=4];
tfbb [label="f(b,b)|term", color=4];
}
{ rank = same;
ta [label="a|term", color=4];
tb [label="b|term", color=4];
}
ff [label="f|function", color=5];
fa [label="a|function", color=5];
fb [label="b|function", color=5];
tfab0 [label="|argument", color=6];
tfab1 [label="|argument", color=6];
tfbb0 [label="|argument", color=6];
tfbb1 [label="|argument", color=6];
t0 -> ta [color=4];
t0 -> tb [color=4];
t1 -> tfab [color=4];
t1 -> tfbb [color=4];
tfab -> ff [color=5];
tfab -> tfab0 [color=6];
tfab0 -> tfab1 [color=7];
tfab0 -> ta [color=8];
tfab1 -> tb [color=8];
tfbb -> ff [color=5];
tfbb -> tfbb0 [color=6];
tfbb0 -> tfbb1 [color=7];
tfbb0 -> tb [color=8];
tfbb1 -> tb [color=8];
ta -> fa [color=5];
tb -> fb [color=5];
% Technical details:
%ff -> ff [dir=forward];
%fa -> fa [dir=forward];
%fb -> fb [dir=forward];
}
\end{frame}
