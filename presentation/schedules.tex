\begin{frame}{Strategies of a theorem prover}
\begin{itemize}[<+->]
\item Automated theorem provers (Vampire, E, etc.) expose large numbers of \emph{parameters}
\item Different \emph{strategies} (configurations) specialize on different input \emph{problems}
\item Given a problem and a time limit, how can we use the allocated time efficiently?
\end{itemize}
\end{frame}

\begin{frame}{Vampire strategies}
%\centering
\only<1>{\includegraphics{figure/presentation/cactus_rel}}
\only<2>{\includegraphics{figure/presentation/cactus_rel_vbs}}
\only<3>{\includegraphics{figure/presentation/cactus_rel_sched}}
\note{\Mi{10000} $\approx$ \SI{5}{\second}}
\end{frame}

%\begin{frame}{Plan of action}
%\begin{enumerate}[<+->]
%\item Strategy discovery: Collect a diverse portfolio of strong strategies
%\item Schedule construction: Construct a strong strategy schedule% using the discovered strategies
%\note{In this work, we focus on static schedules.}
%\end{enumerate}
%\end{frame}

\begin{frame}{Strategy portfolio and performance measurements}
\begin{itemize}
\item \num{1096} \emph{strategies} (configurations of Vampire)
\item \num{7866} \acrlong{fol} \emph{problems} from \acrshort{tptp}
%\item $\num{8621136} = 1096 \cdot 7866$ solver runs
%\item Time to compute: \SI{21}{days} $\times$ \num{120} \gls{cpu} cores
% TODO: How much time was taken by strategy collection and evaluation, respectively?
\end{itemize}

\begin{figure}
\centering
%\includegraphics[clip,trim=0 496px 7066px 0]{figure/presentation/performance_matrix}
\frame{\includegraphics[width=\textwidth]{figure/presentation/performance_matrix}}
% TODO: Make a colored plot with status information. Shades of green: UNS. Shades of red: GUP. Blue: TMO. Separate incomplete strategies.
\end{figure}

%\centering
%\url{https://zenodo.org/records/10814478}
\end{frame}

\begin{frame}{Performance measurements (detail)}
\only<1>{\includegraphics{figure/presentation/heatmap}}
\only<2>{\includegraphics{figure/presentation/cactus_rel_sched}}
\note{We use incomplete strategies - axiom filtering etc.}
\end{frame}

\begin{frame}{Perfect schedule optimization}
\note{Schedule optimization is NP-hard by reduction from the Maximum coverage problem.}
\begin{table}
\centering
\caption{Problems covered out of 7866 for $T = \Mi{256000}$}
\begin{tabular}{@{}l|ll|l@{}}
\toprule
Schedule & Train & Test & Time to construct \\
\midrule
Optimal & \num[round-mode=places,round-precision=1]{6536.7} & \num[round-mode=places,round-precision=1]{6057.4} & $> \SI{16}{\hour}$ (ILP with Gurobi) \\
Greedy & \num[round-mode=places,round-precision=1]{6491.700286078} & \num[round-mode=places,round-precision=1]{6048.924015248} & $< \SI{1}{\minute}$ \\
\bottomrule
\end{tabular}
\end{table}
\note{Gurobi takes more than \SI{24}{\hour} for $T \in \set*{16000, 64000}$.}

\note{On 6293 problems and approximately 876 strategies, Gurobi finds an optimal schedule for $T = 256000$ (relatively easy) in approximately \SI{16}{\hour}.

We may retrieve a feasible sub-optimal solution if we limit the optimization time.}

%\pause
%\emph{How can we prevent overfitting to $\ProblemsTrain$?}
\end{frame}

\definecolor{SliceExtension}{gray}{0}

\begin{frame}{Greedy schedule construction}
\only<1>{
\begin{block}{Slice extension criterion}
\begin{algorithmic}
\State $s, t \gets \argmax_{s \in S, 0 < t \leq T'} \frac{
%\sum_{p \in P} \iverson{\sched{s} < \SolveTimeP{p}{s} \leq \sched{s} + t}
\norm*{\SetBuilder{p \in P'}{\SolveTimeP{p}{s} \leq \sched{s} + t}}^{\textcolor{alpha}{\alpha}}
}{t}$
\end{algorithmic}
\end{block}
}
%\only<2>{\includegraphics{figure/presentation/greedy/0}}
%\only<3>{\includegraphics{figure/presentation/greedy/1}}
%\note{The schedule slices may be ordered using the greedy criterion.}
\end{frame}

\begin{frame}{Regularization for budget \Mi{64000}}
%\begin{figure}
%\raggedleft
\centering
\includegraphics[scale=0.6]{figure/snakes_64000_compact}
% TODO: Prepare an answer: Why are we so far from the main diagonal? Why is the spread so small relatively?
% TODO: Try to plot this graph for a single split. Do we see a similar pattern?
% TODO: Compare to an idealized regularization curve from some paper.
%\end{figure}
\end{frame}
