\subsection{Example: Proving $0 + 2 = 2$}

% Problem in TPTP: "example/022.p"
% Solution generated by Vampire in TSTP: "example/022.s"

\newcommand{\Conj}{C}
\newcommand{\NegConj}{\mli{NC}}
\newcommand{\AxiomS}{A_s}
\newcommand{\AxiomZ}{A_0}
\newcommand{\RulePS}{R_{+s}}
\newcommand{\RuleSP}{R_{s+}}
\newcommand{\RuleN}[1]{R_{#1}}
\newcommand{\rewrite}[1]{\overset{#1}{\longrightarrow}}

\begin{frame}[t]
\frametitle{Example: Proving $2 = 0 + 2$}
\only<1-3>{Axioms for addition on natural numbers:
\begin{align*}
\forall x : \forall y : x + s(y) &= s(x + y)\\
\forall x : x + 0 &= x
\end{align*}
\note{\begin{itemize}
\item $x, y$ are variables. Each equality is implicitly universally quantified.
\item $+$ is a binary addition function.
\item $s$ is a unary successor function.
\item $0$ is a constant.
\item Natural number $n$ is canonically represented by the term $s^n(0)$.
\item Another axiom we could add: $s(x) \neq 0$
\item $\AxiomS$ and $\AxiomZ$ form equational theory $E$.
$C$ is valid in $E$ because it is satisfied by all the models of $E$.
\end{itemize}}
\pause
Conjecture:
\begin{align*}
s(s(0)) &= 0 + s(s(0))
\end{align*}
\pause
How can we prove that the axioms entail the conjecture?}

\only<+(1)->{Negate the conjecture:
\begin{align*}
\forall x : \forall y : x + s(y) &= s(x + y)\\
\forall x : x + 0 &= x\\
s(s(0)) &\neq 0 + s(s(0))
\end{align*}
We attempt to derive a trivial contradiction.

\note{Proof by contradiction:
\begin{itemize}
\item Assume positive premises and negated conjecture.
\item Infer a trivial contradiction.
\end{itemize}}
}

%\only<3>{Clausify (drop universal quantifiers):
%\begin{align*}
%0 + s(s(0)) &\neq s(s(0)) \tag{$\NegConj$}\\
%x + s(y) &= s(x + y) \tag{$\AxiomS$}\\
%x + 0 &= x \tag{$\AxiomZ$}
%\end{align*}
%
%We have normalized the problem to \defn{clause normal form (CNF)}.
%
%%Problem signature: $+$, $s$, $0$
%}
\end{frame}

\begin{frame}[t]{Example: Saturation-based theorem proving}
% Vampire call:
% vampire 022.txt --input_syntax tptp --show_everything on --forward_demodulation off

%1. add(X0,s(X1)) = s(add(X0,X1)) [input]
%2. add(X0,0) = X0 [input]
%3. s(s(0)) != add(0,s(s(0))) [input]
%4. s(s(0)) != s(add(0,s(0))) [superposition 3,1]
%5. s(s(0)) != s(s(add(0,0))) [superposition 4,1]
%6. s(s(0)) != s(s(0)) [superposition 5,2]
%7. $false [trivial inequality removal 6]

\begin{columns}
\begin{column}{0.5\textwidth}
\centering
\textbf{Passive clauses}
\begin{align*}
\visible<1>{\forall x : \forall y : x + s(y) &= s(x + y)}\\
\visible<1-3>{\forall x : x + 0 &= x}\\
\visible<1-2>{s(s(0)) &\neq 0 + s(s(0))}\\
\visible<3-4>{s(s(0)) &\neq s(0 + s(0))}\\
\visible<5>{s(s(0)) &\neq s(s(0 + 0))}\\
\visible<6>{s(s(0)) &\neq s(s(0))}\\
\visible<7>{&\square}\\
\end{align*}
\end{column}
\begin{column}{0.5\textwidth}
\centering
\textbf{Active clauses}
\begin{align*}
\visible<2->{\forall x : \forall y : x + s(y) &= s(x + y)}\\
\visible<4->{\forall x : x + 0 &= x}\\
\visible<3->{s(s(0)) &\neq 0 + s(s(0))}\\
\visible<5->{s(s(0)) &\neq s(0 + s(0))}\\
\visible<6->{s(s(0)) &\neq s(s(0 + 0))}\\
\visible<7->{s(s(0)) &\neq s(s(0))}\\
\end{align*}
\end{column}
\end{columns}
\end{frame}

\begin{frame}[t]{Example: Derivation tree}
\begin{columns}
\begin{column}{0.5\textwidth}
\begin{align*}
\forall x : \forall y : x + s(y) &= s(x + y) \tag{as}\\
\forall x : x + 0 &= x \tag{a0}\\
s(s(0)) &\neq 0 + s(s(0)) \tag{nc}\\
s(s(0)) &\neq s(0 + s(0)) \tag{c4}\\
s(s(0)) &\neq s(s(0 + 0)) \tag{c5}\\
s(s(0)) &\neq s(s(0)) \tag{c6}\\
&\square \tag{f}\\
\end{align*}
\end{column}
\begin{column}{0.5\textwidth}
\centering{\includegraphics[width=\columnwidth,height=\textheight-2cm,keepaspectratio]{figure/presentation/derivation}}
\end{column}
\end{columns}
\end{frame}
