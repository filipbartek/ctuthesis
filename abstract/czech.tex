Automatické dokazování (automatic theorem proving; ATP) pro predikátovou logiku prvního řádu řeší širokou škálu formalizovaných úloh v oblastech, jako jsou matematika a verifikace softwaru. Automatické dokazovače založené na saturaci, zejména vysoce optimalizovaný dokazovač Vampire, představují špičku v oblasti ATP.

Současné moderní dokazovače používají různé heuristiky a jejich úspěšnost na daném problému do značné míry závisí na konfiguraci těchto heuristik. Jejich konfigurace je tradičně doménou expertních uživatelů. Množství dostupných konfigurací a nepředvídatelnost jejich výsledků tento úkol komplikují, a proto je strojové učení (machine learning; ML) slibnou alternativou ke spoléhání se pouze na lidské odborné znalosti.

Tato disertační práce se zabývá aplikací ML v ATP založeném na saturaci, přičemž výzkum se zaměřuje na zvýšení výkonu dokazovače Vampire pomocí tří přístupů. První přístup zahrnuje systém, který pro libovolný vstupní problém doporučí precedenci symbolů pro instanciaci zjednodušujícího uspořádání termů, které efektivně prořezává strom odvozených tvrzení. Druhý přístup využívá systém, který doporučuje váhy symbolů pro schéma výběru klauzulí založené na váženém počítáním symbolů. Nakonec byl ve spolupráci se spoluautory vyvinut systém, který samostatně vyhledává komplementární strategie (konfigurace heuristik) a konstruuje robustní rozvrhy strategií, které dobře generalizují na nové problémy.

Veškerý výzkum prezentovaný v této práci byl publikován v recenzovaných článcích a v práci jsou tyto články uvedeny v jejich konečné publikované podobě.
