Automatické dokazování (ATP) pro predikátovou logiku prvního řádu řeší širokou škálu formalizovaných úloh v oblastech, jako jsou například matematika a verifikace softwaru. Automatické dokazovače založené na saturaci, zejména vysoce optimalizovaný dokazovač Vampire, představují špičku v oblasti ATP.

Současné moderní dokazovače používají různé heuristiky a úspěšnost dokazovače na daném problému do značné míry závisí na konfiguraci těchto heuristik. Konfigurace heuristik je tradičně úkolem expertních uživatelů. Množství dostupných konfigurací a nepředvídatelnost jejich účinků tento úkol komplikují. Proto je strojové učení (ML) slibnou alternativou k lidským odborným znalostem.

Tato disertační práce se zabývá aplikací ML v ATP založeném na saturaci. Výzkum představený v této práci se zaměřuje na zvýšení výkonu dokazovače Vampire pomocí tří přístupů. První přístup je založený na systému, který pro libovolný vstupní problém doporučí precedenci symbolů pro instanciaci zjednodušujicího uspořádání termů, která zajistí efektivní prořezávání grafu odvozených tvrzení. Druhý přístup využívá systém, který doporučuje váhy symbolů pro schéma výběru klauzulí založené na váženém počítáním symbolů. Nakonec byl ve spolupráci se spoluautory vyvinut systém, který samostatně vyhledává komplementární strategie (konfigurace heuristik) a sestavuje z nich robustní rozvrhy, které dobře generalizují na nové problémy.

Veškerý výzkum představený v této práci byl publikován v recenzovaných článcích. V práci jsou tyto články uvedeny v jejich konečné publikované podobě.
