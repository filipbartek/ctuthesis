\chapter{Author's Publications}

%f) a list of the candidate’s publications (projects) including their public acceptance
%(e.g. citations); the list shall be divided into publications related to the dissertation
%topics and other publications; in both sections, the publication shall be subdivided as
%follows: publications in impacted journals, peer reviewed journals, patents, further
%publications excerpted by WoS3 and others; (in case the share of all co-authors is not
%equal, the list shall be completed by the doctoral student’s share of co-authorship for
%individual publications; furthermore, these shares must be supported with a
%declaration of agreement by all the authors).

%f) seznam vlastních publikací (projektů) včetně jejich ohlasů; seznam je členěn
%na publikace vztahující se k tématu disertační práce a na publikace ostatní; v obou
%oddílech se publikace dělí následovně: publikace v impaktovaných časopisech,
%recenzovaných časopisech, patenty, další publikace excerpované WoS a
%publikace ostatní (ke každému článku je přiloženo prohlášení o příspěvku
%jednotlivých spoluautorů (Authorship Statement/Contribution), pokud toto
%prohlášení již není součástí článku dle ŘDS, čl. 7 odst. 3.

% Author contributions:
% https://credit.niso.org/
% https://credit.niso.org/contributor-roles-defined/
% https://www.springer.com/us/editorial-policies/authorship-principles#toc-49265

% Examples:
% https://www.elsevier.com/researcher/author/policies-and-guidelines/credit-author-statement
% https://link.springer.com/journal/11145/submission-guidelines#Instructions%20for%20Authors_Authorship%20principles
% https://onlinelibrary.wiley.com/doi/epdf/10.1087/20150211
% https://www.cell.com/pb/assets/raw/shared/guidelines/CRediT-taxonomy.pdf
% https://pubs.rsc.org/en/content/articlelanding/2023/DD/D2DD00099G

% My publications:
% https://docs.google.com/spreadsheets/d/1rHV33O5apSXxvtCM6_gqnm2dEmj-dV7ij-QQK0__85k

\newcommand{\auth}\emph
\newcommand{\core}[3]{\href{https://portal.core.edu.au/conf-ranks/#3/}{CORE#1 conference rank: #2}}
\newcommand{\wos}[1]{\href{https://www.webofscience.com/wos/woscc/full-record/WOS:#1}{Web of Science record: #1}}
%\newcommand{\citations}[3]{\href{https://scholar.google.com/citations?view_op=view_citation&citation_for_view=DiFzKOgAAAAJ:#3}{Number of citations according to Google Scholar: #1 (excluding self-citations: #2)}}
\newcommand{\scholar}{Source: \href{https://scholar.google.com/}{Google Scholar}. Date extracted: \DTMdisplaydate{2024}{9}{11}{-1}.}
\newcommand{\external}{Excluding self-citations.}
%\newcommand{\citations}[4]{\href{https://scholar.google.com/scholar?cites=#4}{Number of citations\footnote{\scholar{} \external{}}\todo{Join duplicate footnotes.}\todo{Unfiy the date format with the rest of the document.}: #2}}
\newcommand{\citations}[4]{\href{https://scholar.google.com/scholar?cites=#4}{Number of citations\footnote{\scholar{} \external{}}: #2}}

%\Cref{tab:publications} shows an overview of the publications.

%in case the share of all co-authors is not
%equal, the list shall be completed by the doctoral student’s share of co-authorship for
%individual publications

\todo[inline]{Consider removing the table.}
\begin{table}[h]
\begin{ctucolortab}
\centering
\caption{Author's conference and workshop publications.
Each publication is identified by the event (conference or workshop) it was published at.}
\label{tab:publications}
\begin{tabular}{lr|ccrrcc}
%\toprule
\multirow{2}{*}{Event} & \multirow{2}{*}{Year} & \multirow{2}{*}{CORE\tablefootnote{CORE conference rank}} & \multirow{2}{*}{WoS\tablefootnote{Is the publication indexed in Web of Science?}} & \multicolumn{2}{c}{Citations\tablefootnote{\scholar}} & \multirow{2}{*}{Sect.} & \multirow{2}{*}{Bibliography} \\
& & & & All & Ext.\tablefootnote{\external} & & \\
\midrule
\Acrshort{paar}  & 2020 &   &   & 6 & 4 & \ref{sec:results:simple} & \cite{DBLP:conf/cade/Bartek020} \\
\Acrshort{cade}  & 2021 & A & \checkmark & 8 & 5 & \ref{sec:results:npr} & \cite{DBLP:conf/cade/Bartek021} \\
\Acrshort{lpar}  & 2023 & A &   & 3 & 1 & \ref{sec:results:selection} & \cite{DBLP:conf/lpar/Bartek023} \\
\Acrshort{ijcar} & 2024 & A & \checkmark & 3 & 1 & \ref{sec:results:regularization} & \cite{DBLP:conf/ijcar/BartekCS24} \\
\Acrshort{paar}  & 2024 &   &   &   &   & \ref{sec:results:cautious} & \cite{DBLP:conf/paar/BartekC024} \\
%\bottomrule
\end{tabular}
\end{ctucolortab}
\end{table}

\todo[inline]{Add detailed contributions.}
For each of the publications listed below,
the contributions of individual authors follow the \href{https://credit.niso.org/}{CRediT contributor role taxonomy} \cite{DBLP:journals/lp/BrandAAHS15}.
% Role descriptions: https://credit.niso.org/contributor-roles-defined/

\section{Publications Related to the Dissertation Thesis}
% publikace vztahující se k tématu disertační práce

\subsection{Publications Indexed in \href{https://www.webofscience.com/}{Web of Science}}
% další publikace excerpované WoS
\label{sec:wos}

\subsubsection{Neural Precedence Recommender}

\auth{Filip Bártek} and Martin Suda.
Neural Precedence Recommender.
\Acrfull{cade} 2021.
\cite{DBLP:conf/cade/Bartek021}
\\ \core{2021}{A}{918}
%\\ \wos{000693448800030}
\\ \citations{8}{5}{d1gkVwhDpl0C}{4757052500503566147}
\\ See \cref{sec:results:npr}.

\subsubsection{Regularization in Spider-Style Strategy Discovery and Schedule Construction}

%\auth{Filip Bártek}, Karel Chvalovský, and Martin Suda.
%Regularization in Spider-Style Strategy Discovery and Schedule Construction.
%\Acrfull{ijcar} 2024.
%\cite{DBLP:conf/ijcar/BartekCS24}

\begin{itemize}
\item[Authors] \auth{Filip Bártek}, Karel Chvalovský, and Martin Suda
\item[Title] Regularization in Spider-Style Strategy Discovery and Schedule Construction \cite{DBLP:conf/ijcar/BartekCS24}
\item[Conference] \Acrfull{ijcar} 2024\footnote{\core{2023}{A}{1314}}
%\item \wos{001273489700012}
%\item \citations{3}{1}{IjCSPb-OGe4C}{14661423119916603256}
\item[Public acceptance] \citations{3}{1}{IjCSPb-OGe4C}{14661423119916603256}
\end{itemize}

This paper is included in \cref{sec:results:regularization}.

\paragraph{Author contributions.}
Conceptualization: M.S., \auth{F.B.}, and K.C.;
Data curation: \auth{F.B.} and M.S.;
Formal analysis: \auth{F.B.} and M.S.;
Funding acquisition: M.S., \auth{F.B.}, and Josef Urban;
Investigation: \auth{F.B.}, M.S., and K.C.;
Methodology: M.S., \auth{F.B.}, and K.C.;
Project administration: M.S.;
Resources: Josef Urban;
Software: \auth{F.B.}, M.S., and K.C.;
Supervision: M.S.;
Validation: \auth{F.B.};
Visualization: \auth{F.B.} and M.S.;
Writing -- original draft: M.S., \auth{F.B.}, and K.C.;
Writing -- review \& editing: \auth{F.B.}, M.S., and K.C.

\subsection{Other Publications}
% publikace ostatní

\subsubsection{Conference and Workshop Papers}

\begin{enumerate}

\item \auth{Filip Bártek} and Martin Suda.
Learning Precedences from Simple Symbol Features.
\Acrfull{paar} 2020.
\cite{DBLP:conf/cade/Bartek020}
\\ \citations{6}{4}{u-x6o8ySG0sC}{7330041666500653943}
\\ See \cref{sec:results:simple}.

\item \auth{Filip Bártek} and Martin Suda.
How much should this symbol weigh? A \acrshort{gnn}-Advised Clause Selection.
\Acrfull{lpar} 2023.
\cite{DBLP:conf/lpar/Bartek023}
\\ \core{2021}{A}{1596}
\\ \citations{3}{1}{qjMakFHDy7sC}{3677938196667465426}
\\ See \cref{sec:results:selection}.

\item \auth{Filip Bártek}, Karel Chvalovský, and Martin Suda.
Cautious Specialization of Strategy Schedules (Extended Abstract).
\Acrfull{paar} 2024.
\cite{DBLP:conf/paar/BartekC024}
\\ See \cref{sec:results:cautious}.

\end{enumerate}

%\subsubsection{Extended abstracts}

\subsubsection{Datasets}

\begin{enumerate}
\item \auth{Filip Bártek} and Martin Suda.
Vampire strategy performance measurements.
2024.
\cite{bartek10814478}
\end{enumerate}

%\section{Unrelated Publications}
%% publikace ostatní
%\todo[inline]{Should I only include publications created during the PhD? If yes, remove this section.}
%
%\begin{enumerate}
%\item \auth{Filip Bártek}. \foreignlanguage{czech}{Realizace Rabinovy hry na konečných grafech}. Bachelor's thesis. 2012. \cite{Bartek2012thesis}
%\item \auth{Filip Bártek}. Minimum representations of Boolean functions defined by multiple intervals. Master's thesis. 2015. \cite{bartek2015}
%\end{enumerate}
